\documentclass[conference]{IEEEtran}
\IEEEoverridecommandlockouts

% Math packages
\usepackage{amsmath,amssymb,amsfonts}

% Algorithms
\usepackage{algorithmic}

% Figures and graphics
\usepackage{graphicx}       % For including graphics
\usepackage{float}          % For figure placement control
\usepackage{subcaption}     % For subfigures

% Text and fonts
\usepackage{textcomp}       % For special symbols like °

% Colors
\usepackage{xcolor}         % For color definitions

% Tables
\usepackage{booktabs}       % For professional-looking tables

% Bibliography
\usepackage[numbers,sort&compress]{natbib}
\bibliographystyle{unsrtnat}

% Encoding and spacing
\usepackage[utf8]{inputenc} % UTF-8 support
\usepackage{titlesec}       % For customizing section titles
\usepackage{setspace}       % For line spacing control

\definecolor{HKS66}{RGB}{118,185,0}

%verbesserte Datums ausgabe
\usepackage[ddmmyyyy]{datetime}
\renewcommand{\dateseparator}{.}

\begin{document}

\begin{titlepage}
    \thispagestyle{empty}
    \parindent=0pt
    \begin{minipage}[b]{0.35\textwidth}
        \includegraphics[width=\textwidth]{Pictures/htw-berlin_logo.jpg}
    \end{minipage}
    \begin{minipage}[b]{0.65\textwidth}
        ~
    \end{minipage}

    \vspace{0.5em}
    \textcolor{HKS66}{\rule{\linewidth}{0.4mm}}

    \vspace*{\stretch{0.5}}

    % === TITEL UND ART DER ARBEIT ===
    \begin{center}
        {\LARGE\bfseries\color{HKS66} Skriptgesteuerte Erstellung und Konfiguration von Windows-basierten virtuellen Maschinen in Hyper-V mittels PowerShell} \\[1em]
        \textcolor{HKS66}{\rule{\linewidth}{0.4mm}}\\[1.5em]
        {\LARGE\bfseries Bachelorarbeit}
    \end{center}

    \vspace*{\stretch{0.5}}

    % === AUTOR UND HOCHSCHULE ===
    \begin{center}
        \normalsize % (Standardgröße) — du kannst hier z. B. \large draus machen
        von\\[2ex]
        {\bfseries\large\fontsize{14pt}{16pt}\selectfont Kevin Hübner}\\[2ex]
        Matrikelnummer: 570746\\[4ex]
        {\bfseries\fontsize{13pt}{15pt}\selectfont Fachbereich 1: Computer Engineering}\\
        Hochschule für Technik und Wirtschaft Berlin\\[2ex]
    \end{center}

    \vspace*{\stretch{1}}

    % === DATUM ===
    \begin{center}
        {\Large Datum: Berlin, \today}
    \end{center}

    \vspace*{\stretch{1}}

    % === BEGUTACHTER ===
\begin{center}
    \begin{tabular}{ll}
        {\bfseries\large Erstgutachten:} &  \\
        {\bfseries\large Zweitgutachten:} &  \\
    \end{tabular}
\end{center}

    \vspace*{\stretch{2}}

    \textcolor{HKS66}{\rule{\linewidth}{0.4mm}}

\end{titlepage}


\newpage

Theorie:
- PowerShell
- Hyper V 
    - %https://d1wqtxts1xzle7.cloudfront.net/67117145/Benchmarking_the_Performance_of_Microsof20210505-17699-j1sicg.pdf?1738414388=&response-content-disposition=inline%3B+filename%3DBenchmarking_the_Performance_of_Microsof.pdf&Expires=1754378862&Signature=RcQUcGdtIdVThfq~KQwpd1Mhlu1Gi7eJ0IHp-s3QmbQzw~IWOmBXHWxwhm7u-E8WV8BRwc39B13YzHA8qFK0~AdbZrPkfIzpR8uc-3WzoAa4NtFEUx0SMN~N7FYnls8y10jem9~FxcF6TszNB~j9SqSazGu2RjDbm3w09DbYWmmxE~2bJaAUU8cCSCZ9dMxGTVWkRPf68DcdPKp-NTvUdgMtPWVT~2O5SmrmWenthcI73SKH~D8SKrU3WZg8jNI731-B-Q8lxTeYDo8IBpf6J9r9atNdVYycWszi2hE2l0dXdJ8OQ-FBjR9SXcryqfrbkGvD6KhA9PWD0OlpL55NXQ__&Key-Pair-Id=APKAJLOHF5GGSLRBV4ZA
    - two variants hyper v on windows and on windows server
    - requires a processor with hardware assisted virtualizaton
    - based on micro kernelized hypervisors 
        - host os provides drivers
        - virtualizaton runs in the parent partition and has access to hardware devices
        - parent creates child
            - childs have no access to hardware
    -%https://acta.fih.upt.ro/pdf/2016-2/ACTA-2016-2-15.pdf (performance analysis of server system virtualization implemented using hyper v hypervisor)
        - works on the os. 
        - runs on the hardware, under the os
    - % https://www.fujitsu.com/global/documents/about/resources/publications/fstj/archives/vol47-3/paper16.pdf(Windows Server 2008 R2 Hyper-V Server Virtualization)
        - microkernel-based hypervisor 
        - lies above the hardware under the os
        - windows driver stack is embedded in the management os (parent partition)
        - driver compatible with the os can be embedded in the virtualization platform
        - I/O opperations are communications between guest os and management os vis VMBus
        - 
- Sysprep
- WinRM vs WindowsDirect (Powershell WindowsDirect)
Ziel:
- Automatisierung Servererstellung und Konfiguration (Chambionic)
    - Testbenutzer
    - adminbenutzer
    - lokaler admin
    - GPOs
    - Freigaben
    - Datev vorbereitungen (ordner zugriffsrechte)
- zeitersparnis durch skriptgesteuerte Erstellung
- umgehung nerviges durchklicken durch gui elemente
- unnötiges eingeben von einzelnen daten umgehen
- zeitersparnis bei der installation von windows da template

Durchführung:
- VHD(X) vorbereiten
    - Win Server 2025 iso runterladen
    - VM erstellen mit der iso als boot option
    - windows installieren, wie gewohnt (Windows Server 2025 Evaluation Datacenter [Desktopdarstellung])
    - beim festlegen des Adminpasswortes Strg+Shift+F3 -> start in den audit modus
    - unattend.xml unter C:\ ablegen (unattend.xml aus dem ordner "Serverpreparations")
    - sysprep über admin cmd starten mit sysprep /oobe /generalize /shutdown /unattned:filepath
    - vm nicht neustarten!!!
    - vhd(x) der vm in ordner "Serverpreparations" speichern unter dem namen Serverprep.vhdx
- Skript starten
    GUI
    - Abfrage des Speicherortes der zu erstellenden VM
    - Abfrage des Speicherortes der zu importierenden gpo, zu kopierenden vhdx,xml fs rollen installation, xml für die gpo
    - Abfrage der Daten wie Kundenname, OU name, Domain name, NetBios name, Server Ip addressen, passwörter für drms, test user, admin user, standard admin
    - Abfrage des zu verwendenden hyper v switch (netzwerkschnittstelle) 
    - Abfrage der anzahl der kerne für die VMs
    - Abfrage zur RAM Nutzung der VMs
    Skript
        - Aufteilung in Ordner 
        - Main.ps1 ist, wie der Name schon sagt, das Main Skript zum ausführen der Befehle und Skripte 
    - Ordner ActiveDirectoryHandling
        - ADUserGroups -> Create 2 Users and a few standard groups -> puts users into groups
        - OrganizationalUnitStructure -> Create OUs and move severs to correct OU
        - RegistryGroupPolicies -> Create Registry based Grouppolicies and import changed gpo via xml
    - Ordner FileHandling
        - DirectoryPreparations -> initialising a new drive to the fs vm and creating directorys to later share in the network
        - FilehandlingFunctions -> Create file structure, remove sensitive files needed to configure win installation, copy files to VMs
        - Permissions -> handle directory permissions (set access permissions to different adgroups)
    - Ordner VMhandling
        - ChangeIpRenameDc -> disable ipv6, rename Computer and set static ipv4
        - ChangeIpRenameTs -> disable ipv6, rename Computer and set static ipv4
        - VmHandlingFunctions -> ckeck redyness of vm, change vm rescressources, active scripts and set static mac to networkadapter 
    - Ordner UserInterfaceFunctions
        - all userinterface functions to get information from the user
    
- Fehler oder Sonstiges
- Verbesserungen/Erweiterungen/Ideen
    - 

\section{Projektidee und Ausgangssituation}

Die Idee zu diesem Projekt entstand aus der praktischen Arbeit im Bereich Windows-Server-Administration, in dem nahezu ausschließlich mit Windows Server und Hyper-V gearbeitet wird.
In der bisherigen Vorgehensweise erfolgt die Erstellung neuer virtueller Maschinen (VMs) in einem Failover-Cluster (Hyper-V) vollständig manuell – beginnend bei der VM-Anlage bis hin zur vollständigen Konfiguration des Betriebssystems und der benötigten Rollen.
Ein typisches Szenario tritt auf, wenn ein neuer Kunde im Rechenzentrum eingerichtet wird und in der Standardkonfiguration einen Domänencontroller (DC), einen Dateiserver (FS) sowie einen Terminalserver (TS) erhält.
In diesem Fall müssen drei VMs erstellt und jeweils mit Windows Server installiert werden.
Bei kleineren Kunden kann es vorkommen, dass nur eine oder zwei VMs eingesetzt werden, jedoch wird in dieser Arbeit der Drei-VM-Ansatz betrachtet.
Diese Trennung der Rollen hat den Vorteil einer klaren Aufgabenverteilung, einer besseren Übersicht und einer optimierten Lastverteilung zwischen den Systemen.
Die manuelle Einrichtung umfasst zahlreiche wiederkehrende Arbeitsschritte.
Auf dem Dateiserver werden Festplatten für den Speicherplatz angebunden, Ordnerstrukturen angelegt, Freigaben erstellt und NTFS-Berechtigungen vergeben.
Auf dem Terminalserver erfolgt die Installation und Konfiguration der Remotedesktopdienste, wobei im betrachteten Szenario eine sitzungsbasierte Bereitstellung umgesetzt wird.
Dies umfasst die Installation der benötigten Rollenkomponenten wie Connection Broker, Lizenzserver und Session Host.
Zusätzlich wird eine Session Collection für den Kunden angelegt und einer Active-Directory-Gruppe zugewiesen, die für den Zugriff per Remote Desktop autorisiert ist.
Die Lizenzierung muss dabei der geplanten Benutzeranzahl entsprechen.
Eine häufig genutzte Ausnahme stellt die DATEV-Anwendung dar, die in vielen Kundenszenarien eingesetzt wird.
Diese wird aus technischen Gründen meist auf dem Dateiserver installiert, kann jedoch bei größeren Anforderungen auch auf einer dedizierten VM betrieben werden, da sie SQL-Datenbanken mitbringt und entsprechende Ressourcen benötigt.
Die Aktualisierung von DATEV erfordert eine koordinierte Vorgehensweise auf allen beteiligten Systemen und kann zu temporären Verbindungsunterbrechungen führen.
Der Domänencontroller nimmt in diesem Aufbau eine zentrale Rolle ein, da er Benutzer, Gruppen und Gruppenrichtlinien für alle Systeme der Domäne bereitstellt und verwaltet.
Erst nach seiner vollständigen Einrichtung können die anderen VMs der Domäne beitreten, sodass Zugriffsrechte und Gruppenrichtlinien wie vorgesehen greifen.
Zudem bietet er die Möglichkeit, organisatorische Einheiten (OUs) anzulegen, um die Verwaltung von Computern, Servern und Benutzern zu strukturieren.

Die bisherige manuelle Vorgehensweise erfordert eine Vielzahl von Einzelschritten, darunter: Vergabe statischer IP-Adressen, Umbenennung der Rechner, Einbindung zusätzlicher Festplatten, Erstellung von Standard-Benutzern und -Gruppen, Konfiguration von Gruppenrichtlinien sowie Vergabe und Freigabe von Ordnerberechtigungen. Zwar können Gruppenrichtlinien importiert werden, jedoch müssen deren Sicherheitskennungen (SIDs) an die neue Domäne angepasst werden, um die korrekte Funktionsweise sicherzustellen.

Das im Rahmen dieser Arbeit entwickelte PowerShell-Skript automatisiert große Teile dieser wiederkehrenden Aufgaben. Der Benutzer muss nicht mehr durch zahlreiche grafische Dialoge navigieren oder auf manuelle Zwischenschritte warten. Der Automatisierungsgrad ist bewusst so gestaltet, dass nach einer einmaligen Vorbereitung eines Windows-Server-Templates alle weiteren Konfigurationsschritte skriptgesteuert erfolgen können. Das Skript erstellt und konfiguriert drei VMs (DC, FS, TS), richtet die grundlegenden Rollen ein, erstellt Standard-Benutzer und -Gruppen, legt Ordnerstrukturen an, vergibt NTFS-Berechtigungen, setzt statische IP-Adressen und vergibt Computernamen. Das Ergebnis sind einsatzbereite VMs, die lediglich noch aktiviert und im Fall des Terminalservers mit einem Lizenzserver für Remotedesktop-Sitzungen verbunden werden müssen.

Für die Vorbereitung wird zunächst eine VM mit Windows Server installiert und mithilfe des Windows-eigenen Sysprep-Tools generalisiert. Hierfür wird eine vorgefertigte Antwortdatei (\texttt{unattend.xml}) verwendet, die vor der Generalisierung in das System kopiert wird. Nach dem Start im Audit-Modus (über \texttt{Strg + Umschalt + F3}) und dem Platzieren der Antwortdatei im Verzeichnis \texttt{C:\textbackslash Windows\textbackslash System32\textbackslash Sysprep} kann die Generalisierung über den Befehl
\begin{verbatim}
sysprep.exe /oobe /shutdown /generalize /unattend:C:\Windows\System32\Sysprep\unattend.xml
\end{verbatim}
durchgeführt werden. Die resultierende VHDX-Datei (\texttt{Serverprep.vhdx}) dient als Template für die automatisierte Erstellung der zukünftigen VMs.


Skript:


fazit:
Je nach Systemleistung kann so in 20--40 Minuten eine vollständige, funktionsfähige Basisumgebung bereitgestellt werden, die bisher mehrere Stunden manueller Arbeit erforderte.


Verbesserungen:
Das Skript kann noch angepasst werden damit man auch nur 2 oder 1 Server erstellen kann für sehr kleine Kunden. 
Zudem würde eine Anpassung im hinblick auf zusätzlichen Speicher einer VM noch angenehm. Das könnte man ebenfalls über ein GUI lösen.
Weitere Anpassungen am Terminalserver z.B. Lizenzserver konfigurieren etc.
Anpassung des Skripts an den Failovercluster da es bis jetzt nur an einfachen hyper V getestet bzw entwickelt wurde.(vms fehlt eine rolle bzw eigenschafft)



% Refs
\bibliography{ref}
\end{document}

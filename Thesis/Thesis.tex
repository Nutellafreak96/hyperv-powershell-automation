\documentclass[conference]{IEEEtran}
\IEEEoverridecommandlockouts

% Math packages
\usepackage{amsmath,amssymb,amsfonts}

% Algorithms
\usepackage{algorithmic}

% Figures and graphics
\usepackage{graphicx}       % For including graphics
\usepackage{float}          % For figure placement control
\usepackage{subcaption}     % For subfigures

% Text and fonts
\usepackage{textcomp}       % For special symbols like °

% Colors
\usepackage{xcolor}         % For color definitions

% Tables
\usepackage{booktabs}       % For professional-looking tables

% Bibliography
\usepackage[numbers,sort&compress]{natbib}
\bibliographystyle{unsrtnat}

% Encoding and spacing
\usepackage[utf8]{inputenc} % UTF-8 support
\usepackage{titlesec}       % For customizing section titles
\usepackage{setspace}       % For line spacing control

\definecolor{HKS66}{RGB}{118,185,0}

%verbesserte Datums ausgabe
\usepackage[ddmmyyyy]{datetime}
\renewcommand{\dateseparator}{.}

\begin{document}

\begin{titlepage}
    \thispagestyle{empty}
    \parindent=0pt
    \begin{minipage}[b]{0.35\textwidth}
        \includegraphics[width=\textwidth]{Pictures/htw-berlin_logo.jpg}
    \end{minipage}
    \begin{minipage}[b]{0.65\textwidth}
        ~
    \end{minipage}

    \vspace{0.5em}
    \textcolor{HKS66}{\rule{\linewidth}{0.4mm}}

    \vspace*{\stretch{0.5}}

    % === TITEL UND ART DER ARBEIT ===
    \begin{center}
        {\LARGE\bfseries\color{HKS66} Skriptgesteuerte Erstellung und Konfiguration von Windows-basierten virtuellen Maschinen in Hyper-V mittels PowerShell} \\[1em]
        \textcolor{HKS66}{\rule{\linewidth}{0.4mm}}\\[1.5em]
        {\LARGE\bfseries Bachelorarbeit}
    \end{center}

    \vspace*{\stretch{0.5}}

    % === AUTOR UND HOCHSCHULE ===
    \begin{center}
        \normalsize % (Standardgröße) — du kannst hier z. B. \large draus machen
        von\\[2ex]
        {\bfseries\large\fontsize{14pt}{16pt}\selectfont Kevin Hübner}\\[2ex]
        Matrikelnummer: 570746\\[4ex]
        {\bfseries\fontsize{13pt}{15pt}\selectfont Fachbereich 1: Computer Engineering}\\
        Hochschule für Technik und Wirtschaft Berlin\\[2ex]
    \end{center}

    \vspace*{\stretch{1}}

    % === DATUM ===
    \begin{center}
        {\Large Datum: Berlin, \today}
    \end{center}

    \vspace*{\stretch{1}}

    % === BEGUTACHTER ===
\begin{center}
    \begin{tabular}{ll}
        {\bfseries\large Erstgutachten:} &  \\
        {\bfseries\large Zweitgutachten:} &  \\
    \end{tabular}
\end{center}

    \vspace*{\stretch{2}}

    \textcolor{HKS66}{\rule{\linewidth}{0.4mm}}

\end{titlepage}


\newpage

Theorie:
- PowerShell
- Hyper V / Failovercluster
    - %https://d1wqtxts1xzle7.cloudfront.net/67117145/Benchmarking_the_Performance_of_Microsof20210505-17699-j1sicg.pdf?1738414388=&response-content-disposition=inline%3B+filename%3DBenchmarking_the_Performance_of_Microsof.pdf&Expires=1754378862&Signature=RcQUcGdtIdVThfq~KQwpd1Mhlu1Gi7eJ0IHp-s3QmbQzw~IWOmBXHWxwhm7u-E8WV8BRwc39B13YzHA8qFK0~AdbZrPkfIzpR8uc-3WzoAa4NtFEUx0SMN~N7FYnls8y10jem9~FxcF6TszNB~j9SqSazGu2RjDbm3w09DbYWmmxE~2bJaAUU8cCSCZ9dMxGTVWkRPf68DcdPKp-NTvUdgMtPWVT~2O5SmrmWenthcI73SKH~D8SKrU3WZg8jNI731-B-Q8lxTeYDo8IBpf6J9r9atNdVYycWszi2hE2l0dXdJ8OQ-FBjR9SXcryqfrbkGvD6KhA9PWD0OlpL55NXQ__&Key-Pair-Id=APKAJLOHF5GGSLRBV4ZA
    - two variants hyper v on windows and on windows server
    - requires a processor with hardware assisted virtualizaton
    - based on micro kernelized hypervisors 
        - host os provides drivers
        - virtualizaton runs in the parent partition and has access to hardware devices
        - parent creates child
            - childs have no access to hardware
    -%https://acta.fih.upt.ro/pdf/2016-2/ACTA-2016-2-15.pdf (performance analysis of server system virtualization implemented using hyper v hypervisor)
        - works on the os. 
        - runs on the hardware, under the os
    - % https://www.fujitsu.com/global/documents/about/resources/publications/fstj/archives/vol47-3/paper16.pdf(Windows Server 2008 R2 Hyper-V Server Virtualization)
        - microkernel-based hypervisor 
        - lies above the hardware under the os
        - windows driver stack is embedded in the management os (parent partition)
        - driver compatible with the os can be embedded in the virtualization platform
        - I/O opperations are communications between guest os and management os vis VMBus
        - 
- Sysprep
- WinRM vs WindowsDirect (Powershell WindowsDirect)
Ziel:
- Automatisierung Servererstellung und Konfiguration (Chambionic)
    - Testbenutzer
    - adminbenutzer
    - lokaler admin
    - GPOs
    - Freigaben
    - Datev vorbereitungen (ordner zugriffsrechte)
- zeitersparnis durch skriptgesteuerte Erstellung
- umgehung nerviges durchklicken durch gui elemente
- unnötiges eingeben von einzelnen daten umgehen
- zeitersparnis bei der installation von windows da template

Durchführung:
- VHD(X) vorbereiten
    - Win Server 2025 iso runterladen
    - VM erstellen mit der iso als boot option
    - windows installieren, wie gewohnt (Windows Server 2025 Evaluation Datacenter [Desktopdarstellung])
    - beim festlegen des Adminpasswortes Strg+Shift+F3 -> start in den audit modus
    - unattend.xml unter C:\ ablegen (unattend.xml aus dem ordner "Serverpreparations")
    - sysprep über admin cmd starten mit sysprep /oobe /generalize /shutdown /unattned:filepath
    - vm nicht neustarten!!!
    - vhd(x) der vm in ordner "Serverpreparations" speichern unter dem namen Serverprep.vhdx
- Skript starten
    GUI
    - Abfrage des Speicherortes der zu erstellenden VM
    - Abfrage des Speicherortes der zu importierenden gpo, zu kopierenden vhdx,xml fs rollen installation, xml für die gpo
    - Abfrage der Daten wie Kundenname, OU name, Domain name, NetBios name, Server Ip addressen, passwörter für drms, test user, admin user, standard admin
    - Abfrage des zu verwendenden hyper v switch (netzwerkschnittstelle) 
    - Abfrage der anzahl der kerne für die VMs
    - Abfrage zur RAM Nutzung der VMs
    Skript
        - Aufteilung in Ordner 
        - Main.ps1 ist, wie der Name schon sagt, das Main Skript zum ausführen der Befehle und Skripte 
    - Ordner ActiveDirectoryHandling
        - ADUserGroups -> Create 2 Users and a few standard groups -> puts users into groups
        - OrganizationalUnitStructure -> Create OUs and move severs to correct OU
        - RegistryGroupPolicies -> Create Registry based Grouppolicies and import changed gpo via xml
    - Ordner FileHandling
        - DirectoryPreparations -> initialising a new drive to the fs vm and creating directorys to later share in the network
        - FilehandlingFunctions -> Create file structure, remove sensitive files needed to configure win installation, copy files to VMs
        - Permissions -> handle directory permissions (set access permissions to different adgroups)
    - Ordner VMhandling
        - ChangeIpRenameDc -> disable ipv6, rename Computer and set static ipv4
        - ChangeIpRenameTs -> disable ipv6, rename Computer and set static ipv4
        - VmHandlingFunctions -> ckeck redyness of vm, change vm rescressources, active scripts and set static mac to networkadapter 
    - Ordner UserInterfaceFunctions
        - all userinterface functions to get information from the user
    
- Fehler oder Sonstiges
- Verbesserungen/Erweiterungen/Ideen
    - 


    - händisch sind das von unerfahrenen nutzern ca 2h und von erfahrenen nutzern zwischen 1:15 und 1:30
    - auf altem testserver 40 min bestzeit bei großer hitze bis zu 1h

Vorgehensweise:


\section{Projektidee und Ausgangssituation}

Die Idee zu diesem Projekt entstand aus der praktischen Arbeit im Bereich Windows-Server-Administration, in dem nahezu ausschließlich mit Windows Server und Hyper-V gearbeitet wird.
In der bisherigen Vorgehensweise erfolgt die Erstellung neuer virtueller Maschinen (VMs) in einem Failover-Cluster (Hyper-V) vollständig manuell – beginnend bei der VM-Anlage bis hin zur vollständigen Konfiguration des Betriebssystems und der benötigten Rollen.
Ein typisches Szenario tritt auf, wenn ein neuer Kunde im Rechenzentrum eingerichtet wird und in der Standardkonfiguration einen Domänencontroller (DC), einen Dateiserver (FS) sowie einen Terminalserver (TS) erhält.
In diesem Fall müssen drei VMs erstellt und jeweils mit Windows Server installiert werden.
Bei kleineren Kunden kann es vorkommen, dass nur eine oder zwei VMs eingesetzt werden, jedoch wird in dieser Arbeit der Drei-VM-Ansatz betrachtet.
Diese Trennung der Rollen hat den Vorteil einer klaren Aufgabenverteilung, einer besseren Übersicht und einer optimierten Lastverteilung zwischen den Systemen.
Die manuelle Einrichtung umfasst zahlreiche wiederkehrende Arbeitsschritte.
Auf dem Dateiserver werden Festplatten für den Speicherplatz angebunden, Ordnerstrukturen angelegt, Freigaben erstellt und NTFS-Berechtigungen vergeben.
Auf dem Terminalserver erfolgt die Installation und Konfiguration der Remotedesktopdienste, wobei im betrachteten Szenario eine sitzungsbasierte Bereitstellung umgesetzt wird.
Dies umfasst die Installation der benötigten Rollenkomponenten wie Connection Broker, Lizenzserver und Session Host.
Zusätzlich wird eine Session Collection für den Kunden angelegt und einer Active-Directory-Gruppe zugewiesen, die für den Zugriff per Remote Desktop autorisiert ist.
Die Lizenzierung muss dabei der geplanten Benutzeranzahl entsprechen.
Eine häufig genutzte Ausnahme stellt die DATEV-Anwendung dar, die in vielen Kundenszenarien eingesetzt wird.
Diese wird aus technischen Gründen meist auf dem Dateiserver installiert, kann jedoch bei größeren Anforderungen auch auf einer dedizierten VM betrieben werden, da sie SQL-Datenbanken mitbringt und entsprechende Ressourcen benötigt.
Die Aktualisierung von DATEV erfordert eine koordinierte Vorgehensweise auf allen beteiligten Systemen und kann zu temporären Verbindungsunterbrechungen führen.
Der Domänencontroller nimmt in diesem Aufbau eine zentrale Rolle ein, da er Benutzer, Gruppen und Gruppenrichtlinien für alle Systeme der Domäne bereitstellt und verwaltet.
Erst nach seiner vollständigen Einrichtung können die anderen VMs der Domäne beitreten, sodass Zugriffsrechte und Gruppenrichtlinien wie vorgesehen greifen.
Zudem bietet er die Möglichkeit, organisatorische Einheiten (OUs) anzulegen, um die Verwaltung von Computern, Servern und Benutzern zu strukturieren.

Die bisherige manuelle Vorgehensweise erfordert eine Vielzahl von Einzelschritten, darunter: Vergabe statischer IP-Adressen, Umbenennung der Rechner, Einbindung zusätzlicher Festplatten, Erstellung von Standard-Benutzern und -Gruppen, Konfiguration von Gruppenrichtlinien sowie Vergabe und Freigabe von Ordnerberechtigungen. Zwar können Gruppenrichtlinien importiert werden, jedoch müssen deren Sicherheitskennungen (SIDs) an die neue Domäne angepasst werden, um die korrekte Funktionsweise sicherzustellen.

Das im Rahmen dieser Arbeit entwickelte PowerShell-Skript automatisiert große Teile dieser wiederkehrenden Aufgaben. Der Benutzer muss nicht mehr durch zahlreiche grafische Dialoge navigieren oder auf manuelle Zwischenschritte warten. Der Automatisierungsgrad ist bewusst so gestaltet, dass nach einer einmaligen Vorbereitung eines Windows-Server-Templates alle weiteren Konfigurationsschritte skriptgesteuert erfolgen können. Das Skript erstellt und konfiguriert drei VMs (DC, FS, TS), richtet die grundlegenden Rollen ein, erstellt Standard-Benutzer und -Gruppen, legt Ordnerstrukturen an, vergibt NTFS-Berechtigungen, setzt statische IP-Adressen und vergibt Computernamen. Das Ergebnis sind einsatzbereite VMs, die lediglich noch aktiviert und im Fall des Terminalservers mit einem Lizenzserver für Remotedesktop-Sitzungen verbunden werden müssen.

Für die Vorbereitung wird zunächst eine VM mit Windows Server installiert und mithilfe des Windows-eigenen Sysprep-Tools generalisiert. Hierfür wird eine vorgefertigte Antwortdatei (\texttt{unattend.xml}) verwendet, die vor der Generalisierung in das System kopiert wird. Nach dem Start im Audit-Modus (über \texttt{Strg + Umschalt + F3}) und dem Platzieren der Antwortdatei im Verzeichnis \texttt{C:\textbackslash Windows\textbackslash System32\textbackslash Sysprep} kann die Generalisierung über den Befehl
\begin{verbatim}
sysprep.exe /oobe /shutdown /generalize /unattend:C:\Windows\System32\Sysprep\unattend.xml
\end{verbatim}
durchgeführt werden. Die resultierende VHDX-Datei (\texttt{Serverprep.vhdx}) dient als Template für die automatisierte Erstellung der zukünftigen VMs.
---

\section*{Weitere technische Erkenntnisse}

Während der Entwicklung und Umsetzung des Automatisierungsskripts für Hyper-V traten eine Vielzahl technischer Besonderheiten und Fallstricke auf, die bei zukünftigen Projekten berücksichtigt werden sollten. Eine grundlegende Voraussetzung für bestimmte Aktionen, wie beispielsweise den Domänenbeitritt, ist die Verwendung eindeutiger SIDs. Dies wurde durch den Einsatz von \texttt{sysprep} in Verbindung mit einer funktionierenden Antwortdatei sichergestellt. Dabei zeigte sich, dass die Antwortdatei strikt in der Reihenfolge von oben nach unten abgearbeitet wird, weshalb der \texttt{OOBE}-Abschnitt möglichst früh platziert werden sollte.  

Im Bereich der Skripterstellung mit PowerShell war besonders bei der Arbeit mit GUIs und Benutzerinteraktionen Aufmerksamkeit erforderlich. So erwies es sich als sinnvoll, beim Aufbau einer RAM-Auswahl Integer-Werte in Byte-Form anzugeben, da Strings in Arrays nicht zuverlässig verarbeitet wurden. Das Rückgabeverhalten wurde über \texttt{return} angepasst, um korrekte Werte zu liefern. Außerdem zeigte sich, dass in PowerShell die Überprüfung auf \texttt{NULL} konsistent mit der Schreibweise \texttt{NULL -eq \$Variable} erfolgen sollte, um logische Fehler -- insbesondere bei Array-Prüfungen -- zu vermeiden.  

GUI-Fenster verhielten sich in Bezug auf die Anzeige im Vordergrund nicht immer wie erwartet, selbst wenn die Eigenschaft \texttt{TopMost} gesetzt war. Dieses Problem wurde durch die Definition eines Mutterfensters und die gezielte Anzeige untergeordneter Fenster mit \texttt{.Add\_Shown()} gelöst. Passworteingaben ließen sich in der GUI zwar ausblenden, mussten jedoch für die weitere Verarbeitung mittels \texttt{ConvertTo-SecureString -PlainText} in ein SecureString-Format konvertiert werden. Beim Auswählen von Verzeichnissen war zu beachten, dass der Pfad über die Eigenschaft \texttt{InitialDirectory} und nicht \texttt{RootFolder} festgelegt werden muss.  

Für die Eingabevalidierung, beispielsweise bei IP-Adressen, bot sich die Nutzung regulärer Ausdrücke wie \texttt{"[0-9]+\textbackslash.[0-9]+\textbackslash.[0-9]+\textbackslash.[0-9]+"} an. Variablen, die innerhalb von Remote-Jobs oder Sessions genutzt werden, mussten mit dem Präfix \texttt{Using:} übergeben werden -- auch dann, wenn sie zuvor global deklariert wurden. Innerhalb von Strings war eine Variablenersetzung nur mit der Schreibweise \texttt{\$(Variable)} möglich. Bei Abfragen, die einen String erfordern, erwies sich die Verwendung des Parameters \texttt{-ExpandProperty} als notwendig, um Objektrückgaben zu vermeiden.  

Das Ausführen von Befehlen mit dem Parameter \texttt{-AsJob} startete Prozesse im Hintergrund, was problematisch sein konnte, wenn das Ergebnis sofort benötigt wurde. In solchen Fällen war es erforderlich, den Job aktiv zu überwachen, das Ende der Ausführung abzuwarten und anschließend den Job zu entfernen. Beim Einsatz von \texttt{Invoke-Command} konnte entweder direkt mit VM-Namen oder mit zuvor erstellten Sessions gearbeitet werden; letztere mussten nach Abschluss wieder geschlossen werden. Um Anmeldevorgänge zu vermeiden, konnten Anmeldeinformationen als \texttt{PSCredential}-Objekte hinterlegt werden. Diese wurden über \texttt{New-Object} erstellt, wobei das Passwort als SecureString abgefragt und zusammen mit dem Benutzernamen in einer Variablen gespeichert wurde.  

Vor der Ausführung des Skripts wurde geprüft, ob es in einer administrativen Sitzung lief. Falls nicht, wurde es über \texttt{Start-Process} mit dem Verb \texttt{RunAs} neu gestartet. Für eine einheitliche Protokollierung wurde eine Log-Funktion erstellt, die Zeitstempel im Format \texttt{"[dd/MM/yy HH:mm:ss]"} generierte. Nicht benötigte Ausgaben wurden konsequent mit \texttt{| Out-Null} unterdrückt.  

Netzwerkanpassungen innerhalb von Windows-VMs erfolgten mit \texttt{New-NetIPAddress} anstelle von \texttt{Set-NetIPAddress}. Um Dateien vom Hyper-V-Host auf eine VM zu übertragen, musste die Gastdienstschnittstelle aktiviert werden, was über \texttt{Enable-VMIntegrationService} möglich war. Anschließend konnte der Kopiervorgang mit \texttt{Copy-VMFile} durchgeführt werden, wobei Systempfade nur indirekt beschreibbar waren.  

Die Installation bestimmter Serverrollen, wie des Dateiservers, war nur über XML-basierte Konfigurationen möglich. Remote Desktop Services (RDS) konnten vom Domänencontroller auf den Terminalserver installiert werden, wobei zusätzliche Konfigurationen -- etwa Session Collections, RDS-Lizenzierung und Neustarts des Verwaltungsdienstes -- direkt auf dem TS erfolgten.  

Weitere Besonderheiten betrafen die Handhabung von Strings, Hash-Tabellen und VM-Eigenschaften: Beim Splitten eines Strings am Punkt musste der Punkt mit \texttt{"\textbackslash."} escaped werden. VMs konnten vollständig aus Hash-Tabellen heraus erstellt werden, wobei CPU-Anpassungen erst nach der Erstellung und MAC-Adressänderungen nur nach dem ersten Start möglich waren. Neue virtuelle Festplatten wurden per \texttt{Add-VMHardDiskDrive} angebunden, anschließend online geschaltet, benannt, mit einem Laufwerksbuchstaben versehen und formatiert.  

Passwörter von Active-Directory-Konten ließen sich nicht ohne Weiteres zurücksetzen, während lokale Passwörter per Remote- oder PowerShell-Direct-Zugriff problemlos geändert werden konnten. Bestimmte Installationsfehlermeldungen bei RDS ließen sich möglicherweise auf fehlende Lizenzserverkonfigurationen zurückführen.

---

Skript:


Vorgehensweise:
Im Rahmen der Skripterstellung wurde zunächst ein VM-Template erstellt. Diese Einrichtung erfolgt einmalig und setzt die Installation eines Windows-Server-Betriebssystems voraus. Ein wesentlicher Bestandteil dieses Schrittes war die Erstellung einer Antwortdatei (Unattend-XML), die im späteren Prozess die automatisierte Erstkonfiguration ermöglicht.

Zur Generierung der Antwortdatei wurde die Windows-Server-ISO eingebunden und mit dem Windows System Image Manager das darin enthaltene Abbild (install.wim) geladen. Anschließend wurde die Zieledition „Server Standard“ ausgewählt, da diese als Grundlage für die Kunden-VMs vorgesehen ist. Vor der Definition der eigentlichen Antwortdatei wurde eine Katalogdatei erzeugt. Innerhalb der Antwortdatei wurden die relevanten Konfigurationsoptionen integriert, insbesondere im Bereich Microsoft-Windows-Shell-Setup. Dabei lag der Fokus auf dem Abschnitt OOBE, um die Ersteinrichtung zu automatisieren und zusätzliche Benutzerkonten anzulegen.

Nach Erstellung der XML-Datei wurde die Installation so angepasst, dass das Administrator-Passwort über den Audit-Modus (STRG+Shift+F3) definiert werden konnte. Dies ermöglichte individuelle Anpassungen sowie die Absicherung des lokalen Administratorkontos. Für das Skript war ausschließlich relevant, dass die Passwortvergabe gewährleistet, ein zusätzlicher lokaler Administrator angelegt und Einrichtungsdialoge durch die XML-Datei übersprungen wurden. Ergänzend wurde das Tastaturlayout auf Deutsch gesetzt.

Die vollständige XML-Datei wurde im Verzeichnis des Windows-Systemvorbereitungstools (Sysprep, i.d.R. unter C:\Windows\System32\Sysprep) gespeichert. Anschließend erfolgte die Ausführung von Sysprep mit den Parametern /oobe, /generalize, /shutdown sowie /unattend:(Pfad zur XML-Datei). Dadurch wurde die Windows-Installation generalisiert und von der spezifischen VM-Umgebung entkoppelt. Die einmalige Erstellung und Konfiguration beanspruchte auf einem leistungsfähigen System etwa fünf bis zehn Minuten.

Durch die Generalisierung der Windows-Installation wird erreicht, dass die virtuelle Festplatte (VHD/X) einer erstellten VM kopiert und mehrfach wiederverwendet werden kann, ohne dass Konflikte zwischen den Sicherheits-IDs (SIDs) der Administratorbenutzer verschiedener VMs auftreten. Entscheidend ist hierbei, dass die betreffende VM nach der Generalisierung nicht erneut gestartet wird, bevor die VHD(X)-Datei kopiert wurde, da ansonsten die Generalisierung ihre Gültigkeit verliert.

Im nächsten Schritt erfolgte die Erstellung von VMs mithilfe von PowerShell. Der grundlegende Ablauf entspricht dabei den Prozessen, die auch aus anderen Virtualisierungslösungen bekannt sind: Jedem virtuellen System werden ein Name, ein Speicherort, Arbeitsspeicher, CPU-Kerne, ein Netzwerkadapter sowie eine Bootfestplatte zugewiesen. Da bereits eine vorbereitete VHDX-Datei für die Bootpartition vorliegt, konnte dieser Teil übersprungen werden. Für den Fileserver wurde zusätzlich eine separate Festplatte eingerichtet, die der Speicherung der Nutzerdaten dient und entsprechend in das System eingebunden wird. Die Konfiguration über PowerShell erforderte somit lediglich Schritte, die auch über die grafische Oberfläche ausgeführt werden könnten.

Das vorbereitete VHD(X)-Template wird an den vorgesehenen Speicherort kopiert, anschließend erfolgt die Zuweisung von Ressourcen wie Arbeitsspeicher und Prozessoranzahl. Nach Abschluss dieser Konfiguration ist die VM startbereit und kann für Tests verwendet werden. Für jede neue VM wird dieser Prozess identisch durchgeführt, wodurch sich manuelle Vorgehensweisen mit den PowerShell-basierten Methoden vergleichen lassen.

Vor dem Einstieg in weitere Schritte war eine zusätzliche Anpassung hinsichtlich der Netzwerkadapter erforderlich. Da die Systeme über TeamViewer verwaltet werden, musste für den Netzwerkadapter jeder VM eine statische MAC-Adresse vergeben werden. Dies verhindert, dass die Systeme aus der Geräteliste in TeamViewer verschwinden. Um die MAC-Adresse zuweisen zu können, wird die VM einmalig gestartet, sodass Windows eine temporäre Adresse generiert. Anschließend wird die Maschine heruntergefahren und in Hyper-V die MAC-Adresse auf statisch gesetzt. Danach kann der reguläre Arbeitsablauf fortgesetzt werden.

Perfekt, ich sehe worauf du hinauswillst 👍. Ich habe auch diesen Text so umformuliert, dass er wissenschaftlich-nüchtern, aus der Perspektive einer Prozessbeschreibung wirkt, gleichzeitig aber weiterhin dein Vorgehen beschreibt. Das Ziel ist, dass es wie ein kohärenter Abschnitt in einer wissenschaftlichen Arbeit klingt, nicht wie eine Schritt-für-Schritt-Anleitung.
Wissenschaftlich überarbeitete Fassung

Zur Verwaltung der Kunden-VMs wurde eine einheitliche Ordnerstruktur implementiert. Hierfür wurde ein eigenes Skript (FilehandlingFunctions) entwickelt, das für jeden Kunden automatisch identische Verzeichnisse anlegt, wobei die Struktur jeweils unter dem individuellen Kundennamen eingetragen wird. Innerhalb dieser Hierarchie werden die vorbereiteten VHD(X)-Dateien aus dem zentralen Vorbereitungsverzeichnis in die entsprechenden Kundenordner kopiert. Vor der Erstellung überprüft das Skript, ob die betreffende Struktur bereits vorhanden ist, um redundante Duplikate zu vermeiden. Die Umsetzung basiert auf einfachem File Handling, wobei neue Verzeichnisse mit dem PowerShell-Befehl New-Item erzeugt werden.

Neben der Verzeichnisstruktur wird für jeden Kunden zusätzlich ein Logfile eingerichtet. Dieses ermöglicht sowohl die Nachverfolgung des Skriptablaufs als auch die Identifikation möglicher Fehlerquellen oder Prozessunterbrechungen. Da das VHD(X)-Template für die VM-Erstellung standardmäßig die Datei unattend.xml enthält, in der Administratorpasswörter hinterlegt sind, entsteht ein nicht zu vernachlässigendes Sicherheitsrisiko. Um dies zu vermeiden, wird die unattend.xml nach erfolgreicher Erstellung der Kundenordner sowie der zugehörigen VM mithilfe einer im Skript implementierten Funktion wieder gelöscht.

Bei der Entwicklung des Skripts wurde zudem der Umgang mit Variablen und deren Gültigkeitsbereichen berücksichtigt. Variablen aus dem Hauptskript können zwar in ein Unterskript übernommen werden, müssen dann jedoch mit dem Präfix Using gekennzeichnet werden. Effizienter ist es, Funktionen mit Parametern zu verwenden, wodurch sich das Vorgehen strukturell mit der Trennung von Header- und Code-Dateien in C/C++ vergleichen lässt – allerdings ohne die Existenz tatsächlicher Header-Dateien. Funktionen werden über ein separates Skript eingebunden, sodass sie entweder lokal wiederverwendet oder per Invoke-Command auf einem Zielsystem remote ausgeführt werden können.

Mithilfe von PowerShell Direct ist es möglich, innerhalb von Hyper-V direkt Befehle in den VMs auszuführen, ohne auf Netzwerkverbindungen angewiesen zu sein. Alternativ können Befehle über das Netzwerk mittels Windows Remote Management (WinRM) ausgeführt werden, wobei jedoch die Problematik des sogenannten Kerberos Double Hop auftreten kann. Vor der Installation von Rollen ist es notwendig, den VMs statische IP-Adressen zuzuweisen und sowohl dem Fileserver als auch dem Terminalserver den Domain Controller als DNS-Server einzutragen. Darüber hinaus erfolgt eine Anpassung der Computernamen zur eindeutigen Identifikation. Für diesen Zweck wurden spezifische kleine Skripte (ChangeIpRenameTs bzw. ChangeIpRenameDc) entwickelt, die die Konfiguration automatisieren.

Die Einrichtung des Domain Controllers basiert ebenfalls auf PowerShell. Bei einer manuellen Installation lässt sich vor der Heraufstufung ein Skript generieren, das die durchgeführten Arbeitsschritte abbildet. Dieses Vorgehen bietet wertvolle Einblicke in die notwendigen Befehle, zusätzlich stehen umfassende Informationen in der offiziellen Microsoft-Dokumentation zur Verfügung. Für die automatisierte Konfiguration wurde eine Hashmap erstellt, die alle erforderlichen Parameter wie Domänenname, NetBIOS-Name, DSRM-Passwort sowie Pfade für System- und Logdateien enthält. Anschließend wurde die benötigte Rolle installiert und der Domain Controller mittels Install-ADDSForest heraufgestuft. Nach einem Neustart stand damit die Grundkonfiguration zur Verfügung. Vor dem Neustart wurde zudem der Google-DNS-Server (8.8.8.8) in die Liste der Weiterleitungen aufgenommen.

Im Gegensatz dazu gestaltet sich die Installation der Fileserver-Rolle komplexer, da es sich hierbei um eine Unterrolle handelt. Eine direkte Konfiguration über PowerShell ist nur eingeschränkt möglich. Stattdessen können jedoch XML-Dateien verwendet werden, die automatisch aus einer manuellen Installation generiert werden. Da die Rolleninstallation auf allen Systemen identisch erfolgt und keine zusätzlichen spezifischen Einstellungen erforderlich sind, wurde dieses Vorgehen als praktikable Lösung genutzt.

Für die erfolgreiche Automatisierung war es erforderlich, dass bestimmte XML- und Konfigurationsdateien auf den Zielsystemen verfügbar sind. Um dies zu ermöglichen, musste die Gastdienstschnittstelle innerhalb der VMs aktiviert werden, wodurch sich Dateien direkt vom Hostsystem in die Gastsysteme übertragen lassen. Vor dem eigentlichen Transfer wurde in jeder VM ein Ordner angelegt, in den die relevanten Dateien kopiert werden. Für den Domain Controller waren dies Gruppenrichtlinien-Dateien, während für den Fileserver die Installations-XML der entsprechenden Rolle vorgesehen war. Die Installation der Fileserver-Rolle erfolgt gemeinsam mit der Konfiguration einer statischen IP-Adresse, da diese Rolle unabhängig vom Domain Controller eingerichtet wird (DeployFileServerRole).

Zum betrachteten Zeitpunkt standen bereits drei VMs zur Verfügung, von denen zwei mit den vorgesehenen Rollen ausgestattet waren und ein funktionsfähiger Domain Controller bereitgestellt war. Der darauffolgende Schritt bestand darin, die verbleibenden Server in die Domäne aufzunehmen, nachdem der Domain Controller vollständig gestartet war. Dies konnte mit einer dedizierten PowerShell-Funktion erreicht werden, die unter Verwendung von Invoke-Command den Domänenbeitritt initiiert.

Im Zuge der Experimente mit den VM-Templates zeigte sich jedoch, dass Fehler in der Generalisierung (z. B. durch fehlerhafte Unattend-Dateien) dazu führen können, dass lokale Administratoren auf unterschiedlichen Systemen identische Sicherheits-IDs (SIDs) erhalten. In diesem Fall schlägt der Domänenbeitritt fehl. Daher ist bei der Erstellung der Unattend-Datei besondere Sorgfalt erforderlich. Die Überprüfung der SIDs kann nach der VM-Erstellung über den Befehl Get-LocalUser -Name Administrator | FL erfolgen. Werden identische SIDs festgestellt, ist davon auszugehen, dass die Generalisierung mit Sysprep nicht korrekt durchgeführt wurde, was wiederum zum Abbruch des Skripts führen würde.

Nach erfolgreichem Domänenbeitritt der Systeme folgte die Vorbereitung des Fileservers. Hierzu wurde die zusätzliche Festplatte zunächst online geschaltet und mit dem Partitionsstil GPT initialisiert. Anschließend wurde die gesamte Speicherkapazität in einer Partition zusammengefasst, die mit einem Laufwerksbuchstaben (im Skript: D:) und einer eindeutigen Bezeichnung („Daten“) versehen wurde. In diesem Verzeichnis wurden standardisierte Basisordner angelegt, die später als Grundlage für Freigaben dienen. Da Freigaben mit bestimmten Active-Directory-Gruppen verknüpft sind, war zuvor eine grundlegende AD-Struktur zu etablieren.

Die erstellte Active-Directory-Struktur umfasste exemplarische Benutzerkonten (z. B. einen Testbenutzer und einen Administrationsaccount), mehrere Standardgruppen (beispielsweise Scan_LW, Datevuser, Daten_LW, GF für Geschäftsführung) sowie Organisationseinheiten zur strukturierten Trennung von Benutzern, Gruppen und Computern. Die Benutzer wurden in Form von Hashmaps angelegt, während Gruppen über den Befehl New-ADGroup erstellt wurden. Im Anschluss wurden die Benutzer den entsprechenden Gruppen zugeordnet.

Zur Grundstruktur des Active Directory gehören auch Standardrichtlinien, etwa für Netzlaufwerke, Remotedesktop-Einstellungen, Einschränkungen der Eingabeaufforderung und Skriptausführung sowie das Deaktivieren von „New Outlook“. Da viele dieser Gruppenrichtlinien auf Registry-Einträgen basieren, können sie mithilfe von Hashmaps umgesetzt werden. Die benötigten Registry-Keys lassen sich entweder durch das Auslesen bestehender Richtlinien über Get-GPPrefRegistryValue oder durch Dokumentationen im Internet identifizieren.

Komplexer gestaltet sich die Abbildung von Netzlaufwerken. Die manuelle Konfiguration eines Netzlaufwerks führt zu einer Vielzahl an Registry-Einträgen. Werden diese als Grundlage für eine neue Gruppenrichtlinie übernommen, schlägt die Umsetzung häufig fehl, sodass die Laufwerke nicht im Explorer angezeigt werden. Dadurch werden sie für Endnutzer unbrauchbar. Um dieses Problem zu umgehen, wurde eine bereits funktionierende Gruppenrichtlinie mit den Netzlaufwerken aus einem bestehenden System exportiert, in die VM des Domain Controllers übertragen, an die SID der neuen Domäne angepasst und anschließend importiert.

Die importierten Gruppenrichtlinien (GPOs) verhalten sich nach der Anpassung der SIDs wie manuell erstellte Richtlinien. Ein direkter Eingriff in einzelne Registry-Einträge ist damit nicht erforderlich, da die XML-Dateien der GPOs in einer Schleife verarbeitet und die enthaltenen Gruppen angepasst werden können. Mit Abschluss dieser Schritte stand das Grundgerüst des Active Directory bereit, womit lediglich noch wenige Arbeitsschritte bis zur Fertigstellung des Automatisierungsskripts zur Serverstruktur erforderlich waren.

Um den Zugriff auf die Verzeichnisse des Terminalservers zu ermöglichen, wurden die Ordner zunächst freigegeben und mit den notwendigen Berechtigungen versehen. Die Freigabe konnte mit dem Befehl New-SmbShare umgesetzt werden, wobei mithilfe von Hashmaps und Schleifen Freigabename, Pfad und Berechtigungen automatisiert zugewiesen wurden. Im Anschluss wurden die NTFS-Berechtigungen auf Verzeichnis- und Dateiebene konfiguriert. Hierfür wurden die bestehenden Rechte eines Verzeichnisses zunächst mit Get-Acl in einer Variablen gespeichert, anschließend über eine Hashmap neue Einträge definiert (Benutzer bzw. Gruppe, Berechtigungsumfang sowie Anwendungsbereich: Ordner, Unterordner und Dateien). Diese wurden in ein neues ACL-Objekt überführt und mit Set-Acl auf das jeweilige Verzeichnis angewendet.

Ein Problem dieser Vorgehensweise besteht darin, dass Rechte jeweils nur für einen Ordner und eine Gruppe gleichzeitig gesetzt werden können. Der Versuch, mehrere Einträge parallel zu übernehmen, führte entweder zu fehlerhaften Berechtigungen oder zu fehlenden Fehlermeldungen, sodass die Rechtevergabe stets einzeln vorgenommen werden musste. Nach der Umsetzung der Freigaben und NTFS-Berechtigungen verblieb als letzter Bestandteil die Einrichtung des Terminalservers.

Die Installation der dazugehörigen Rollen konnte nicht vollständig über PowerShell-Kommandos innerhalb des Gastesystems erfolgen. Stattdessen wurden die Rollen mithilfe von PowerShell Direct über den Hyper-V-Host remote in die VM installiert und anschließend konfiguriert. Damit konnte eine funktionsfähige Terminalserver-Umgebung bereitgestellt werden, wenngleich die Lizenzierungskonfiguration nicht Teil des Skripts war. Auch ohne diese Konfiguration ist eine Anmeldung mittels Remotedesktop für einzelne Benutzer möglich, wodurch Kernfunktionen des Terminalservers zur Verfügung stehen.

Zum Abschluss wurden sämtliche VMs einmalig neu gestartet, um einen einheitlichen Betriebszustand herzustellen und alle Konfigurationen sowie Änderungen, die einen Neustart erforderten, gültig zu machen. Lediglich beim Terminalserver besteht eine geringe Wahrscheinlichkeit, dass ein Dienst nach dem Neustart nicht ordnungsgemäß ausgeführt wird. Da im Anschluss an die Grundkonfiguration jedoch weitere Installationen und Anpassungen erfolgen, wird dieses Restrisiko als vernachlässigbar eingestuft.

---
Test:

\chapter{Testumgebungen}

Zur Validierung des entwickelten Skripts wurden Testdurchläufe in zwei unterschiedlichen Umgebungen durchgeführt. Dadurch kann die Stabilität und Portabilität der Lösung besser eingeschätzt werden, da sowohl die eingesetzte Hardware als auch die Software variieren.

\section{Hardware- und Softwarekonfiguration}

\begin{table}[H]
\centering
\caption{Hardware- und Softwarekonfiguration der Testumgebungen}
\label{tab:hardware_config}
\begin{tabular}{|l|l|l|}
\hline
\textbf{Komponente} & \textbf{Testumgebung Zuhause} & \textbf{Testumgebung Server} \ \hline
CPU & AMD Ryzen 7 7800X3D, 8 Kerne / 16 Threads, 4,3--4,5GHz & Intel Xeon E3-1230 v5, 4 Kerne / 8 Threads, 3,7GHz \ \hline
RAM & 32GB DDR5-6000 & 64GB DDR4-2133 (4 $\times$ 16GB) \ \hline
Speicher & NVMe-SSD, 2TB, PCIe 3.0, 3500MB/s Lesen, 3000MB/s Schreiben & SSD-RAID (Fujitsu PRAID EP400i, SCSI-Interface) \ \hline
PowerShell-Version & 7.5.2 (Core Edition) & 5.1 \ \hline
Windows-Version & Windows 11 Pro & Windows Server 2025 Standard \ \hline
\end{tabular}
\end{table}

\section{Virtuelle Maschinen}

Die virtuellen Maschinen wurden in beiden Testumgebungen identisch konfiguriert:

\begin{itemize}
\item \textbf{CPU-Kerne:} 2
\item \textbf{RAM:} 2GB pro VM
\item \textbf{Zusätzliche VHDX:} 35MB auf dem File Server
\end{itemize}

\chapter{Testmethodik}

Zur Überprüfung des entwickelten Skripts wurden mehrere Testdurchläufe auf den beschriebenen Umgebungen durchgeführt. Ziel war die Validierung der Funktionsfähigkeit, Stabilität und Ausführungsdauer des Automatisierungsprozesses. Der Fokus lag dabei nicht auf einer detaillierten funktionalen Prüfung der konfigurierten Dienste, sondern auf der wiederholbaren und automatisierten Erstellung der virtuellen Maschinen.

\section{Vorgehensweise}

\begin{itemize}
\item Das Skript wurde in beiden Umgebungen jeweils \textbf{fünfmal} ausgeführt.
\item Während der Ausführung wurde das integrierte Logging verwendet, um die Dauer der Testläufe zu protokollieren.
\item Fehler und Auffälligkeiten im Ablauf wurden dokumentiert.
\end{itemize}

\section{Bewertungskriterien}

Die Testdurchläufe wurden anhand folgender Kriterien bewertet:

\begin{itemize}
\item \textbf{Ausführungsdauer:} Zeit, die das Skript für Erstellung und Grundkonfiguration der VMs benötigt.
\item \textbf{Stabilität:} Fähigkeit des Skripts, aufeinanderfolgende Durchläufe fehlerfrei zu absolvieren.
\item \textbf{Reproduzierbarkeit:} Konsistenz der Ergebnisse zwischen den Durchläufen und Umgebungen.
\end{itemize}

\section{Zielsetzung}

Die Testmethodik dient primär der Validierung des Automatisierungsprozesses und nicht einer funktionalen Überprüfung der konfigurierten Rollen und Dienste. Mehrfache Durchläufe ermöglichen die Erfassung von Abweichungen in der Ausführungsdauer und die Bewertung der Zuverlässigkeit des Skripts.

\chapter{Testergebnisse}

\section{Testumgebung Zuhause}

\begin{table}[h!]
\centering
\caption{Zeitübersicht der Testdurchläufe – Home (relative Zeiten)}
\label{tab:home_times}
\begin{tabular}{|l|l|l|l|l|l|}
\hline
\textbf{Schritt} & \textbf{Log 1} & \textbf{Log 2} & \textbf{Log 3} & \textbf{Log 4} & \textbf{Log 5} \ \hline
Start & 15:44:20 & 17:30:04 & 18:16:06 & 18:47:11 & 20:53:45 \
Kopieren fertig & 0:00:26 & 0:00:31 & 0:00:32 & 0:00:31 & 0:00:32 \
VMs erstellt & 0:00:30 & 0:00:36 & 0:00:36 & 0:00:34 & 0:00:36 \
VMs angepasst & 0:00:31 & 0:00:36 & 0:00:36 & 0:00:35 & 0:00:37 \
VMs gestartet & 0:00:34 & 0:00:39 & 0:00:39 & 0:00:37 & 0:00:40 \
Windows initialisierung fertig & 0:01:13 & 0:01:34 & 0:01:30 & 0:01:28 & 0:01:32 \
... & ... & ... & ... & ... & ... \
\textbf{Gesamtzeit} & 0:18:40 & 0:18:43 & 0:19:10 & 0:18:33 & 0:19:04 \ \hline
\end{tabular}
\end{table}

\section{Testumgebung Server}

\begin{table}[H]
\centering
\caption{Zeitübersicht der Testdurchläufe – Server (relative Zeiten)}
\label{tab:server_times}
\begin{tabular}{|l|l|l|l|l|l|}
\hline
\textbf{Schritt} & \textbf{Log 1} & \textbf{Log 2} & \textbf{Log 3} & \textbf{Log 4} & \textbf{Log 5} \ \hline
Start & 09:24:01 & 08:46:27 & 09:42:02 & 11:24:36 & 12:37:53 \
Kopieren fertig & 0:06:02 & 0:06:03 & 0:04:53 & 0:04:53 & 0:04:52 \
VMs erstellt & 0:06:15 & 0:06:14 & 0:05:03 & 0:05:04 & 0:05:04 \
VMs angepasst & 0:06:15 & 0:06:14 & 0:05:04 & 0:05:04 & 0:05:04 \
VMs gestartet & 0:06:20 & 0:06:19 & 0:05:09 & 0:05:09 & 0:05:09 \
Windows initialisierung fertig & 0:10:52 & 0:12:12 & 0:09:52 & 0:11:17 & 0:10:21 \
... & ... & ... & ... & ... & ... \
\textbf{Gesamtzeit} & 0:43:11 & 0:43:29 & 1:09:41 & 0:43:08 & 0:41:37 \ \hline
\end{tabular}
\end{table}

\chapter{Fazit und Ausblick}

Im Rahmen dieser Arbeit wurde ein PowerShell-basiertes Automatisierungsskript entwickelt, das die Erstellung und Konfiguration von Windows-basierten virtuellen Maschinen in einer Hyper-V-Umgebung ermöglicht.
Die Ergebnisse der durchgeführten Testdurchläufe bestätigen die Funktionsfähigkeit und Stabilität des Skripts. Die Tests erfolgten in zwei unterschiedlichen Umgebungen: einer leistungsstarken Desktop-Konfiguration mit AMD Ryzen 7 7800X3D, 32 GB RAM und NVMe-SSD sowie einem Server mit Intel Xeon E3-1230 v5, 64 GB RAM und SSD-RAID. Die VMs wurden in beiden Umgebungen identisch mit 2 CPU-Kernen, 2 GB RAM und einer zusätzlichen 35 MB VHDX-Datei konfiguriert.

Die Testdurchläufe zeigten, dass die Bereitstellung der Basisumgebung in der Desktop-Umgebung innerhalb von etwa 18 bis 19 Minuten abgeschlossen werden konnte. Auf dem Server variierte die Dauer aufgrund der unterschiedlichen Hardware und PowerShell-Version zwischen ca. 41 und 69 Minuten, wobei die meisten Durchläufe um 43 Minuten lagen. Die Unterschiede erklären sich vor allem durch die I/O-Leistung des Speichersystems, die CPU-Auslastung und die PowerShell-Version.

Während der Tests traten lediglich kleinere Fehlermeldungen auf, beispielsweise im Zusammenhang mit der Einrichtung der Session Collection oder gelegentlichen Abbrüchen von Remoteverbindungen. Diese hatten jedoch weder in den Testumgebungen mit Domänenanbindung noch im Betrieb mit Testbenutzern spürbare Auswirkungen auf die Funktionalität. Insgesamt zeigten die mehrfachen Durchläufe eine hohe Reproduzierbarkeit und Stabilität des Skripts.

Trotz des erreichten Funktionsumfangs bestehen mehrere Potenziale für zukünftige Verbesserungen. Hierzu zählt die Möglichkeit, für kleinere Kundenumgebungen nur ein oder zwei Server bereitzustellen, wodurch Ressourcen und Zeit eingespart werden könnten. Eine flexible Anpassung des der virtuellen Maschine zugewiesenen Speichers – idealerweise über eine grafische Benutzeroberfläche (\textit{GUI}) – würde zudem die Bedienbarkeit und den Einsatzbereich erweitern. Auch die vollständige Automatisierung zusätzlicher Terminalserver-Konfigurationen, wie etwa die Lizenzserver-Einrichtung, könnte den manuellen Administrationsaufwand weiter reduzieren.

Für den Einsatz in hochverfügbaren Umgebungen wäre eine Anpassung an Failover-Cluster-Architekturen erforderlich. Dies erfordert die Erweiterung des Skripts um clusterrelevante Rollen und Eigenschaften, da es bislang ausschließlich in Einzelhost-Szenarien getestet wurde. Ein weiteres Optimierungspotenzial liegt in der Ablösung der derzeitigen statischen Wartezeit nach der Installation des Domänencontrollers durch ein dynamisches, ereignisgesteuertes Verfahren, um die Bereitstellungsdauer zu verkürzen und die Professionalität der Umsetzung zu erhöhen. Darüber hinaus könnte die Benutzeranlage durch den Import vordefinierter XML-Dateien automatisiert werden, was die Konsistenz und Wiederholbarkeit der Konfiguration verbessern würde.

Insgesamt zeigt die Arbeit, dass durch den gezielten Einsatz von PowerShell-Skripten in Verbindung mit Hyper-V eine deutliche Effizienzsteigerung in der Serverbereitstellung erreicht werden kann. Die durchgeführten Tests auf unterschiedlichen Hardwarekonfigurationen belegen dabei die Flexibilität und Anpassungsfähigkeit des Skripts. Die skizzierten Verbesserungsansätze bieten darüber hinaus ein hohes Potenzial für zukünftige Entwicklungen, insbesondere im Hinblick auf Flexibilität, Skalierbarkeit und Integration in komplexere IT-Infrastrukturen.
\newpage

Text:\\

Idee:
Automatisierung einer Serverstruktur in Hyper-V. Dabei wird berücksichtigt, dass es einen Domain Controller, einen Fileserver und einen Terminalserver gibt. Die Erstellung soll außerdem das Anlegen von freigegebenen Ordnern und Gruppenrichtlinien sowie die Einrichtung eines Domain Admins und eines Test Users für die Firma beinhalten. Zudem soll die Installation von Windows durch ein VM-Template umgangen werden.

Stand in der Firma:
Derzeit wird alles manuell über die GUI in Windows erledigt. Zusätzlich wird auf jeder VM Windows separat installiert. Selbst Gruppenrichtlinien, die man per PowerShell erstellen kann, oder das Anlegen von Usern und Gruppen werden über die GUI durchgeführt. Das ist insbesondere heutzutage unprofessionell, da Windows zahlreiche Möglichkeiten zur Automatisierung bietet.

Vorgehensweise:
Der erste Schritt für das Skript ist die Erstellung eines VM-Templates. Dieses wird einmalig manuell erstellt, und dabei muss Windows Server nur ein einziges Mal installiert werden. Wichtig ist außerdem die Erstellung einer Antwortdatei (unattend.xml) im XML-Format. Diese lässt sich problemlos mit der ISO von Windows Server und dem Windows-eigenen Windows System Image Manager erzeugen. Dazu mountet man die ISO und lädt im Windows System Image Manager das Image (install.wim) aus der ISO. Anschließend wählt man die gewünschte Version von Windows Server, zum Beispiel Server Standard oder Server Datacenter. Für die Kunden-VMs wählen wir Server Standard. Bevor wir eine Antwortdatei erstellen können, muss zunächst eine Katalogdatei erzeugt werden. Sobald diese vorliegt, kann die Antwortdatei erstellt werden.

Alle relevanten Optionen finden sich im Windows Image Fenster unter „Microsoft-Windows-Shell-Setup“. Wichtig ist vor allem der Unterpunkt „OOBE“, um viele Schritte bei der Ersteinrichtung von Windows zu überspringen. Zusätzlich lässt sich dort auch ein weiterer lokaler Benutzer im Abschnitt „UserAccounts“ anlegen. Um eine Option in die Antwortdatei einzufügen, wählt man per Rechtsklick den passenden Teil der Antwortdatei aus. Dies ist wichtig, da OOBE verschiedene Phasen hat, in denen unterschiedliche Konfigurationen greifen. Den OOBE-Abschnitt können wir nur unter oobeSystem einordnen, daher haben wir hier keine Wahl.

Tests:
Beim testen der Geschwindigkeit auf verschiedenen Maschinen wird die VM mit 2 Kernen und 2GB ram erstellt.

% Refs
\bibliography{ref}
\end{document}

\documentclass[conference]{IEEEtran}
\IEEEoverridecommandlockouts

% Math packages
\usepackage{amsmath,amssymb,amsfonts}

% Algorithms
\usepackage{algorithmic}

% Figures and graphics
\usepackage{graphicx}       % For including graphics
\usepackage{float}          % For figure placement control
\usepackage{subcaption}     % For subfigures

% Text and fonts
\usepackage{textcomp}       % For special symbols like °

% Colors
\usepackage{xcolor}         % For color definitions

% Tables
\usepackage{booktabs}       % For professional-looking tables

% Bibliography
\usepackage[numbers,sort&compress]{natbib}
\bibliographystyle{unsrtnat}

% Encoding and spacing
\usepackage[utf8]{inputenc} % UTF-8 support
\usepackage{titlesec}       % For customizing section titles
\usepackage{setspace}       % For line spacing control

\definecolor{HKS66}{RGB}{118,185,0}

%verbesserte Datums ausgabe
\usepackage[ddmmyyyy]{datetime}
\renewcommand{\dateseparator}{.}

\begin{document}

\begin{titlepage}
    \thispagestyle{empty}
    \parindent=0pt
    \begin{minipage}[b]{0.35\textwidth}
        \includegraphics[width=\textwidth]{Pictures/htw-berlin_logo.jpg}
    \end{minipage}
    \begin{minipage}[b]{0.65\textwidth}
        ~
    \end{minipage}

    \vspace{0.5em}
    \textcolor{HKS66}{\rule{\linewidth}{0.4mm}}

    \vspace*{\stretch{0.5}}

    % === TITEL UND ART DER ARBEIT ===
    \begin{center}
        {\LARGE\bfseries\color{HKS66} Skriptgesteuerte Erstellung und Konfiguration von Windows-basierten virtuellen Maschinen in Hyper-V mittels PowerShell} \\[1em]
        \textcolor{HKS66}{\rule{\linewidth}{0.4mm}}\\[1.5em]
        {\LARGE\bfseries Bachelorarbeit}
    \end{center}

    \vspace*{\stretch{0.5}}

    % === AUTOR UND HOCHSCHULE ===
    \begin{center}
        \normalsize % (Standardgröße) — du kannst hier z. B. \large draus machen
        von\\[2ex]
        {\bfseries\large\fontsize{14pt}{16pt}\selectfont Kevin Hübner}\\[2ex]
        Matrikelnummer: 570746\\[4ex]
        {\bfseries\fontsize{13pt}{15pt}\selectfont Fachbereich 1: Computer Engineering}\\
        Hochschule für Technik und Wirtschaft Berlin\\[2ex]
    \end{center}

    \vspace*{\stretch{1}}

    % === DATUM ===
    \begin{center}
        {\Large Datum: Berlin, \today}
    \end{center}

    \vspace*{\stretch{1}}

    % === BEGUTACHTER ===
\begin{center}
    \begin{tabular}{ll}
        {\bfseries\large Erstgutachten:} &  \\
        {\bfseries\large Zweitgutachten:} &  \\
    \end{tabular}
\end{center}

    \vspace*{\stretch{2}}

    \textcolor{HKS66}{\rule{\linewidth}{0.4mm}}

\end{titlepage}


\newpage

Theorie:
- PowerShell
- Hyper V 
- Sysprep
- WinRM vs WindowsDirect (Powershell WindowsDirect)
Ziel:
- Automatisierung Servererstellung und Konfiguration (Chambionic)
    - Testbenutzer
    - adminbenutzer
    - lokaler admin
    - GPOs
    - Freigaben
    - Datev vorbereitungen (ordner zugriffsrechte)

Durchführung:
- VHD(X) vorbereiten
    - Win Server 2025 iso runterladen
    - VM erstellen mit der iso als boot option
    - windows installieren, wie gewohnt (Windows Server 2025 Evaluation Datacenter [Desktopdarstellung])
    - beim festlegen des Adminpasswortes Strg+Shift+F3 -> start in den audit modus
    - unattend.xml unter C:\ ablegen (unattend.xml aus dem ordner "Serverpreparations")
    - sysprep über admin cmd starten mit sysprep /oobe /generalize /shutdown /unattned:filepath
    - vm nicht neustarten!!!
    - vhd(x) der vm in ordner "Serverpreparations" speichern unter dem namen Serverprep.vhdx
- Skript starten
    GUI
    - Abfrage des Speicherortes der zu erstellenden VM
    - Abfrage des Speicherortes der zu importierenden gpo, zu kopierenden vhdx,xml fs rollen installation, xml für die gpo
    - Abfrage der Daten wie Kundenname, OU name, Domain name, NetBios name, Server Ip addressen, passwörter für drms, test user, admin user, standard admin
    - Abfrage des zu verwendenden hyper v switch (netzwerkschnittstelle) 
    - Abfrage der anzahl der kerne für die VMs
    - Abfrage zur RAM Nutzung der VMs
    Skript
        - Aufteilung in Ordner 
        - Main.ps1 ist, wie der Name schon sagt, das Main Skript zum ausführen der Befehle und Skripte 
    - Ordner ActiveDirectoryHandling
        - ADUserGroups -> Create 2 Users and a few standard groups -> puts users into groups
        - OrganizationalUnitStructure -> Create OUs and move severs to correct OU
        - RegistryGroupPolicies -> Create Registry based Grouppolicies and import changed gpo via xml
    - Ordner FileHandling
        - DirectoryPreparations -> initialising a new drive to the fs vm and creating directorys to later share in the network
        - FilehandlingFunctions -> Create file structure, remove sensitive files needed to configure win installation, copy files to VMs
        - Permissions -> handle directory permissions (set access permissions to different adgroups)
    - Ordner VMhandling
        - ChangeIpRenameDc -> disable ipv6, rename Computer and set static ipv4
        - ChangeIpRenameTs -> disable ipv6, rename Computer and set static ipv4
        - VmHandlingFunctions -> ckeck redyness of vm, change vm rescressources, active scripts and set static mac to networkadapter 
    - Ordner UserInterfaceFunctions
        - all userinterface functions to get information from the user
    
- Fehler oder Sonstiges
- Verbesserungen/Erweiterungen/Ideen
    - 

% Refs
\bibliography{ref}
\end{document}
